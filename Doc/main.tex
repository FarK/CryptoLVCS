\documentclass[a4paper,12pt]{article}
\usepackage[utf8]{inputenc}
\usepackage[spanish]{babel}
\usepackage{amsmath}
\usepackage{import}
\usepackage{graphicx}
\usepackage{subcaption}
\usepackage[spanish]{varioref}
\usepackage{xparse}
\usepackage{appendix}
\usepackage{color}
\usepackage[spanish,linesnumbered]{algorithm2e}
\usepackage{hyperref}
\hypersetup{colorlinks=true,
	linkcolor=blue,
	citecolor=blue,
	filecolor=blue,
	urlcolor=blue,
	pdftitle= Esquema de criptografía visual basado en letras,
	pdfauthor= David Solís Martín - Carlos Falgueras García,
	pdfsubject= Esquema de criptografía visual basado en letras,
	pdfkeywords= VCS  DVCS  PVCS  LVCS
}

\def\Title{Esquema de criptografía visual basado en letras}
\def\Authors{David Solís Martín \\ Carlos Falgueras García}
%\def\Tutor{}

%Referencia capítulos, secciones, etc; con número y título
\newcommand{\refcont}[1]{\ref{#1}-\nameref{#1}}
\newcommand{\reffig}[1]{Figura \ref{#1}}

%Espaciado entre párrafos
\newlength{\parseparation}
\setlength{\parseparation}{1.5em}  %Distancia entre párrafos por defecto
\newcommand{\parset}[1]{\setlength{\parskip}{#1}}  %Comando para setear distancia
\newcommand{\parreset}{\setlength{\parskip}{\parseparation}}  %Comando para dejarla como estaba
\parreset

%Configuración algoritmos pseudocódigo
\RestyleAlgo{boxed}


%%%%%%%%%%
% MACROS %
%%%%%%%%%%

\begin{document}
\subimport{Title/}{Title}

\tableofcontents
\newpage

%Imports de las partes
\subimport{Introduccion/}{Introduccion}
\newpage
\subimport{VCS/}{VCS}
\newpage
\subimport{LVCS/}{LVCS}
\newpage
\subimport{Implementacion/}{Implementacion}
\newpage
\subimport{Aplicaciones/}{Aplicaciones}
\newpage

%Bibliografía
\section{Bibliografía}
\nocite{*}
\bibliographystyle{plain}
\bibliography{bibliografia.bib}

%Anexos
\newpage
\section{Anexos}
\subsection{Repositorio}
Esta es la dirección del repositorio público Git que contiene tanto el código
del programa, como el código \LaTeX\ de la documentación.

\textbf{Dirección del repositorio:} \url{https://github.com/FarK/CryptoLVCS}

\end{document}
